\section{Matrix Transpose}
\label{sec:transpose}

Abstract matrix transpose can be performed ``en masse'' using set comprehension.  The CSR transpose algorithm, on the other hand, involves nested loops with dependencies, shown in \figurename~\ref{algo:transpose}.

\begin{figure}
TRANSPOSE PSEUDOCODE
\caption{CSR Transpose}
\label{algo:transpose}
\end{figure}

Have modeled transpose using a method similar to the \texttt{kpos} method used in ELL to CSR translation that does, but have also demonstrated a way to generate a table of iteration steps -- still in development and method is being refined.

The algorithm consists of four distinct steps: (1) counting of columns, (2) calculation of pointers, (3) copying of values, and (4) shifting of pointers.  An array, $iao$, evolved continuously through this process and is used in all four steps.  We are mostly concerned with the third step, as it contains an inner loop that has a dependency on the outer loop, seen in Algorithm~\ref{algorithm:transpose}.

% \LinesNumbered
\begin{algorithm}
\SetStartEndCondition{ }{}{}
\SetKwProg{Fn}{def}{\string:}{}
\SetKwFunction{Range}{range}
\SetKw{KwTo}{in}\SetKwFor{For}{for}{\string:}{}
\SetKwIF{If}{ElseIf}{Else}{if}{:}{elif}{else:}{}
\SetKwFor{While}{while}{:}{fintq}
\newcommand{\forcond}[2]{$#1$ \KwTo\Range{$#2$}}
\SetAlgoNoEnd
\DontPrintSemicolon
\For{\forcond{i}{n}}{
  \For{\forcond{k}{ia[i], ia[i+1]}}{
    j = ja[k]\;
    nxt = iao[j]\;
    ao[nxt] = a[k]\;
    jao[nxt] = i\;
    iao[j] = next + 1\;
  }
}
\BlankLine
\caption{The copy values step in CSR transpose}
\label{algorithm:transpose}
\end{algorithm}
% \LinesNotNumbered

The first and fourth steps make use of set comprehensions, the second step uses a common modeling paradigm to generate values, but the third step requires the final method introduced in this paper.  Rather than generate a sequence of integers, as in kpos method, generate a sequence of arrays.  This method demonstrated in the \texttt{copyvals} predicate in \figurename~\ref{model:transpose}.  May move this method of generating a sequence of arrays to be included with the kpos method, depending on the final form of the table-of-iterations method.

\begin{figure}
\lstinputlisting[language=alloy]{models/transpose.als}
\caption{CSR Transpose}
\label{model:transpose}
\end{figure}