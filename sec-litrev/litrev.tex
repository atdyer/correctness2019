\section{Related Work}
\label{sec:litrev}

Arnold et al.~\cite{arnold} design a functional language and proof method for the automatic verification of sparse matrix codes.  Their ``little language" (LL) can be used to specify sparse codes as functional programs in which computations are sequences of high-level transformations on lists.  These models are then automatically translated into Isabelle/HOL where they can be verified for full functional correctness.  The authors use this automated proof method to verify a number of sparse matrix formats and their sparse matrix-vector multiplication operations.  Their approach is purely functional, relying on typed $\lambda$-calculus and Isabelle libraries to perform proofs using Isabelle/HOL.  LL, a shallow embedding in Isabelle/HOL, provides simple, composable rules that can be used to fully describe a sparse matrix format, removing the burden of directly writing proofs.  In quantifying rule reuse, the authors find that ``on average, fewer than 19\% of rules used by a particular format are specific to this format, while over 66\% of these rules are used by at least three additional formats, a significant level of reuse.  Even of the rules needed for more complex formats (CSC and JAD), only up to a third are format-specific.''~\cite{arnold}  The method enables a significant amount of rule reuse, but does not entirely prevent one from having to write some amount of $\lambda$-calculus.  Filling in these gaps may prove difficult for those without a strong background in functional programming and theorem proving.  Our approach, on the other hand, may appeal to an audience of scientists and engineers, who are accustomed to working with models anyway, and with the kind of automatic, push-button analysis supported by Alloy, those who develop software can focus on modeling and design instead of theorem proving.

Arnold approach does not support modeling of assembly of sparse matrices.  Relational approach in combination with state change allows us to model the assembly process as well as updates to matrix values.  Relational approach also allows for modeling of format conversions.