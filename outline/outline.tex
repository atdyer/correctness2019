\documentclass{article}

\parindent 0pt
\parskip 10pt

\usepackage[margin=1in]{geometry}
\usepackage[us,12hr]{datetime}
\usepackage{color}
\usepackage[leftbars,xcolor]{changebar}
\usepackage{hyperref}

\title{Bounded Verification of\\Sparse Matrix Computations}
\author{Tristan, Alper, John}
\date{\today\ at \currenttime}

\begin{document}

\maketitle

\begin{abstract}
\parindent 0pt
\parskip 10pt

Show how to model and reason about BOTH structure and behavior, can
model rich state and operations, sparse matrix representations and
algorithms on them.

Use Alloy, a declarative modeling language with a SAT solver,
automatic proofs within a scope.

Show how to perform tests for refinement and equivalence, the form of
the ``proof obligations'' (need a better phrase).

Also introduce a new Alloy idiom for state changes, for bounded
iteration and changing state.  Other contributions?
\end{abstract}

\section{Introduction}

-sparse matrices:

Sparse matrix computations are central to many applications in
scientific computing ... or say avoiding zeros by not assembling,
etc., but must be dealt with.  Also sparse matrices for big data,
machine learning, right?

Mention common *representations*, used historically, libraries
available, but also new representations being created to take
advantage of new hardware characteristics (cite), and new
architectures like GPUs (cite).

Mention common *operations* on sparse matrices, primarily sparse
matrix vector multiplication for solving linear systems and
eigenvalues, etc.

-our particular interest:

Our own interest arose when optimizing ADCIRC, a popular large-scale
ocean model, trying to assure that changes preserved correctness.
Long run times, testing is hard.

First use (by our group?) for subdomain modeling, can make incremental
model changes and simulate at an incremental computational cost.
Huge performance gains.  Maybe show an image of a subdomain?

Using Alloy, a declarative language with automatic analysis, can
simulate and check ... not a common application for state-based formal
methods ... (predicate abstraction type things?)

Continuing work on internal algorithms, solvers (and the particular
representations they require) ... e.g., ADCIRC uses itpack-v and
ellpack format ... other solvers CSR

Assembly from meshes works directly on the data structures for
performance, sometimes hard to ensure correct in all cases, corner
cases not missed.

{\color{red} Alper, this is done for performance reasons, right?
  Simple setter/getter functions could be used to hide internal
  representation of a sparse matrix, but guess (?) it's difficult to
  create operations like that that don't compromise performance?}

-scope and organization of paper:

...

\section{Alloy and Bounded Verification (or ``Approach'')}

Uses a SAT solver, automatic, no reals.

Small scope hypothesis

Figure: cube in 3 axes (Alloy) ... 3 axes and random dots (testing)

\cbcolor{green}
\cbstart
Unit testing and test driven development... aim to reveal bugs by testing individual components, incomplete and less effective at testing combinations of components

Libraries provide nice abstractions on structure and behavior, but often still require a lot of work to be done on the developer's side, and to maintain performance may not perform checks like one might assume. Great example of this here: \url{https://github.com/scipy/scipy/issues/8778} where we see scipy does not check the representation invariant of a csr matrix, specifically for performance reasons.

Another possibly useful example... rep invariant is maintained but the internal sorting is reversed... unintentional side effects may go unnoticed, some solvers require ordering to be maintained... \url{https://github.com/scipy/scipy/issues/9639}
\cbend

Implicitness in spec ... Alloy ``generates'' what might be viewed as
input/data for operations ... arbitrary sparse matrices

IMPORTANT: for numerical software, hard to perform ``parts''
of a large-scale simulation by simply extracting code ... have to
create ``valid'' data for input, populate data structures ... here we
say properties required of data and Alloy generates that for us ... so
in some ways easier than testing a part of a large-scale program.

\subsection*{Alloy for structure and behavior}

Not obvious for numerical software, probably.

\cbcolor{green}
\cbstart
Importance of proper specification to which we are designing? An example here of a discussion about the `correct' behavior when storing explicit zeros in a representation: \url{https://github.com/scipy/scipy/issues/3343}
\cbend

Structure:

- Class-like sigs for modeling rich state\\
- But weak support for Ints (and no reals)

On the latter, two approaches ... for some problems predicate
abstaction is sufficient (cf. subdomain modeling), for others, may
just need to rely on numbers being ``distinct'' (the latter is the
case for sparse matrices):

From Jackson's text: To figure out whether integers are necessary, ask
yourself what properties are actually relied upon. For example, a
communication protocol that numbers its messages may RELY ONLY ON THE
NUMBERS BEING DISTINCT; or it may rely on them increasing; or perhaps
even being totally ordered.

Behavior:

- Since class-like features are based on relations, we have a
relational language, declarative, but it has navigation expressions
(generalizing dot as relational join) that make data modeling natural,
as well as first-order predicate calculus + transitive closure, a rich
modeling language.

- Of course numerical programs are written in imperative languages
like Fortran, C++.

**We can avoid overspecifying order of computations, and when we need
an ordering can impose it ... usually simply.  Alloy has an idiom,
adding a ``time'' column to a relation.  Later we describe a
complementary approach that works well for matrices and bounded (needs
to be bounded?) iteration.

\section{Sparse Matrix Representations in Alloy}

Basic representation of zero and non-zero values.

\cbcolor{green}
\cbstart
\begin{displaymath}
Value = \{Zero, Value_0, Value_1, \ldots, Value_n\}
\end{displaymath}
Also, a diagram showing inheritance...Value is a signature, Zero is an extension of Value.
\cbend

Signatures for CSR (just, or also ELL?).

Hoare checks of an ADT for correctness (or earlier?) ... sound
  ... commuting diagram (general, figure) ... math expression for the
  refinement check (here, or next section?)

Abstraction function and representation invariant, define what the
terms mean and present them for CSR.

\section{Sparse Matrix Operations in Alloy}

\cbcolor{green}
\cbstart
The three methods we've used to model structured loops... one simple one and two more complex that introduce time-varying state:

(1) set comprehensions for simple cases that are easier to reason about -- ``en masse'', 

(2) method used in ellcsr that makes use of the 'some' quantifier to populate a sequence of loop values (kpos) -- for semi-structured looping needing time-varying state... 

(3) method of using a signature to represent internal loop state, for cases in which internal looping structure is more complicated (time-varying state again), arrays are incrementally modified, as in transpose...

Description of each method with a simple example, maybe the counting of even numbers models?

Now the following sections make use of each of these methods in order of growing complexity... MVM has no time-varying state, translation needs only the first time-varying state method, transpose makes use of all three methods.
\cbend

\cbcolor{yellow}
\cbstart
Declarative languages need a way to express time-varying state.

Variety of idioms particular to the language, for lazy functional
languages recursively defined infinite
sequences~\cite{hughes1989functional}.
\cbend

\section{Matrix-Vector Multiplication}

SumProd, representation of expressions ... sums and products, zeros
and non zeros, commutative ... rep that's easy to use and check ... a
lightweight approach, not Seigel's completely general approach.

Also no time-varying state here ... again, simple ...

nature of the check ... a triangle, show in figure and in math and Alloy


\section{Translation Between Formats}

Ellpack to CSR

We say it in intro (?), but could elaborate on why of interest to us
... use of PARDISO and other solvers in ADCIRC that require CSR
... ADCIRC assembles into ELL ... can reimplement assembly or
translate between ELL and CSR.

\cbcolor{green}
\cbstart
Other solvers we'd like to use... ITPACK and Pardiso use CSR, cuSPARSE uses BCSR, Armadillo uses CSC.

Have an algorithm from SPARSKIT, want to determine if it can be performed in-place.
\cbend

Here we have time-varying state (kpos variable) ... introduce a new
idiom: bounded iteration with time-varying state, has a conditional so
you can't ``pre-compute'' values ... do this using an array + stuttering
... sort of like util/ordering in Alloy, but variable is an array
indexed by ``time'' ticks.

\section{Matrix Transpose}

Algorithm in pseudo-code and definition in Alloy.

Sould be simple? (in the sense of an ``en masse'' description without
state updates ... no need for ``time varying'' state).

\cbcolor{green}
\cbstart
Abstract transpose is ``en masse'', CSR is not quite as simple. Looking like all three methods described above will be used...
\cbend

Show the nature of the refinement check in Alloy.

After we check this, can point out it's really CSC, define the
abstraction function for CSC, and do another check with its
abstraction function.

\section*{Boundary Conditions in ADCIRC? (probably not)}

Documentation says symmetric (we checked it) ...

\cbcolor{green}
\cbstart
We thought the zeroing of rows was incorrectly modifying values that need to be saved before the zeroing of columns occurs... turns out that is not that case and the order is correct.
\cbend

\cbcolor{red}
\cbstart
We also thought columns weren't supposed to be zeroed out, and they
are ... can check, we did.  (I know Alper already knows rows and
columns are both zeroed out).
\cbend

\section{Dicussion ... or ``Testing''?}


What kinds of bugs can be found ... ``mutation'' testing?

I wonder to what extent our Alloy models can catch off-by-one or other
indexing problems.  Is it possible some get missed?  I mean, for
transpose, for instance, an error would have to show up in the
corresponding abstract matrix (via alpha, the abstraction function)
for our approach to "notice" it.  Since Alloy predicates are "total"
(they produce sort of "null" results, empty sets, when things go
wrong), the effects may be "visible" in the abstract matrix, or maybe
some can be missed ... maybe?

\section{Related Work}

a. Arnold - Isabelle/HOL ... ``full functional correctness'' (a side
     goal), looking at compilers and optimization, created a little
     functional language

from paper:

Before settling on the design presented in this paper, we set as our
goal the full functional verification of imperative sparse code, in
the style presented in Section 2.1. However, even the simple CSR
format turned out to be rather overwhelming. We attempted to verify
its correctness in multiple ways: (i) manually with Hoare-style logic,
both with first-order predicates and inductive predicates; (ii) with
ESC/Java [6]; (iii) with TVLA [13]; and (iv) using a SAT-based BOUNDED
MODEL CHECKER. The results were unsatisfactory either because it took
weeks to develop the necessary invariants (i, ii), the abstraction was
too complex for us to manage (iii), or BECAUSE THE CHECKER SCALED
POORLY (iv). Eventually, we concluded that we needed to verify sparse
codes at a higher level of abstraction (and separately compile the
verified code into efficient low-level code). Turning our attention to
functional programs allowed us to replace explicit loops over arrays
with maps and a few fixed reductions over lists, which in turn
simplified the formulation and encapsulation of inductive invariants.

\cbcolor{green}
\cbstart
Rule reuse was significant, but still requires knowledge of lambda-calculus. Does not support modeling of assembly process.
\cbend

b. Kotlyar -

A relational approach to the compilation of sparse matrix programs

\cite{kotlyar1997relational}

From Stodghill's dissertation

A RELATIONAL APPROACH TO THE AUTOMATIC GENERATION OF SEQUENTIAL SPARSE
MATRIX CODES

The idea of using viewing sparse computations as relational queries
from which joins could be discovered and scheduled using query
optimization was first suggested by Kotlyar [81].

Libraries are one approach to alleviating the user s burden but often
libraries do not give the user sufficient control over algorithmic and
storage format decisions Also the natural abstraction layer between
the user s code and the library code often serves as a barrier to
important policy and optimization decisions It seems that instead of
having the sparse implementation provided a priori what is needed is a
tool that will generate portions of the sparse implementation from the
dense specification . We call such a tool a sparse compiler.

c. Siegal - expression tables for checking equivalence (types of
           equivalence: Herbrand, IEEE, and real)

it's not sparse matrices, the symbolic handling relevant

d. Our work

No other sparse matrix work of ours, but any use of Alloy is relevant,
would like to include the KeYmaera X work if we can say why.

Alper's KeYmaera X work

A little more about subdomain modeling?

other past work?


\section{Conclusions}

and future work:

a. newer, more complex formats, hybrid formats

``A hybrid format for better performance of sparse matrix-vector
multiplication on a GPU'' (cite) (I like this article because it shows
that improvements to formats are still being made, and they're
becoming more complex, I guess.  Also, the perspective on ELL and CSR
is good.  By Gropp.

b. include mesh representations in Alloy to model assembly, to optimize

c. complexities of GPU problems

\bibliographystyle{IEEEtran}
\bibliography{outbib}

\end{document}





